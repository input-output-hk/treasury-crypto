\usepackage[utf8]{inputenc}
\usepackage{graphicx}
\graphicspath{ {./images/} }
\usepackage{enumitem}
\usepackage{url}
\usepackage{nameref}

\usepackage{placeins} % put this in your pre-amble
\usepackage{flafter}

%\usepackage{draftwatermark}
%\SetWatermarkText{DRAFT}
%\SetWatermarkScale{3}

\usepackage[dvipsnames]{xcolor}
\usepackage{hyperref}
\newcommand\myshade{85}
\colorlet{mylinkcolor}{violet}
\colorlet{mycitecolor}{YellowOrange}
\colorlet{myurlcolor}{Aquamarine}
\hypersetup{
    breaklinks=true,
	colorlinks=true, %set true if you want colored links
	linktoc=all,     %set to all if you want both sections and subsections linked
    linkcolor=violet,
    filecolor=magenta,      
    urlcolor=cyan,
    citecolor=YellowOrange
}

\usepackage{amsthm}
\theoremstyle{definition}
\newtheorem{definition}{Definition}[section]
\usepackage{geometry}
\newgeometry{vmargin={24mm}, hmargin={28mm,28mm}}

\usepackage{listings}
\lstset{
  language=Java,
  aboveskip=-2mm,
  belowskip=-2mm,
  showstringspaces=false,
  columns=flexible,
  ={\small\ttfamily},
  numbers=none,
  breaklines=true,
  breakatwhitespace=true,
  tabsize=4,
  keywordstyle=\bfseries,
  morekeywords={type, Inputs, Outputs}
}

\usepackage{titlesec}
\usepackage{titletoc}
\usepackage[title,titletoc]{appendix}
\renewcommand{\appendixname}{Appendix}
\usepackage{caption}
\usepackage{array,tabularx}
\newenvironment{conditions}
  {\par\vspace{\abovedisplayskip}\noindent
   \tabularx{\columnwidth}{>{$}l<{$} @{${}-{}$} >{\raggedright\arraybackslash}X}}
  {\endtabularx\par\vspace{\belowdisplayskip}}
\usepackage[skins]{tcolorbox}
\newtcolorbox{protocolframe}[2][]{%
  enhanced,colback=white,colframe=black,coltitle=black,
  sharp corners,boxrule=0.4pt,
  fonttitle=\itshape,
  attach boxed title to top left={yshift=-0.3\baselineskip-0.4pt,xshift=2mm},
  boxed title style={tile,size=minimal,left=0.5mm,right=0.5mm,
    colback=white,before upper=\strut},
  title=#2,#1
}
\newcommand{\subsubsubsection}[1]{\paragraph{#1}\mbox{}\\}
\setcounter{secnumdepth}{4}
\setcounter{tocdepth}{4}

% Dependencies from the treasury paper

\usepackage{amssymb,amsfonts}
\usepackage{amsmath}
\newcommand{\ignore}[1]{}
\usepackage[normalem]{ulem}
\usepackage{color}

%\geometry{letterpaper,top=1.5in,bottom=1.5in,left=1in,right=1in}


\newtheorem{theorem}{Theorem}
\newtheorem{lemma}{Lemma}
\newtheorem{proposition}{Proposition}
\newtheorem{corollary}{Corollary}
\newtheorem{example}{Example}
\newtheorem{remark}{Remark}

\newtheorem{observation}{Observation}
\newtheorem{assumption}{Assumption}
\newtheorem{fact}{Fact}


\newcommand{\change}[1]{\textcolor{red}{#1}}
\newcommand{\remove}[1]{\textcolor{red}{\sout{#1}}}

\newcommand{\Com}{\mathsf{Com}}
\newcommand{\Opt}{\mathsf{Opt}}


% Mathematical notation
\newcommand{\NN}        {\mathbb{N}}
\newcommand{\ZZ}        {\mathbb{Z}}
\newcommand{\PP}        {\mathbb{P}}
\newcommand{\FF}        {\mathbb{F}}
\newcommand{\abs}  [1]  {\left|#1\right|}
\newcommand{\set}  [1]  {\left\{#1\right\}}
\newcommand{\floor}[1]  {\lfloor #1\rfloor}
\newcommand{\ceil} [1]  {\lceil #1\rceil}
\newcommand{\pr}   [1]  {\Pr \left[#1\right]}
\newcommand{\poly}      {\operatorname{poly}}
\newcommand{\polylog}   {\operatorname{polylog}}
\newcommand{\Oh}        {\mathcal{O}}
\newcommand{\param}     {\mathsf{param}}

% Common mathematical operators
\newcommand{\iseq}{\stackrel{\scriptscriptstyle ?}{=}}
\newcommand{\getsu}{\gets_{\scriptscriptstyle \hspace*{-0.7em}u\hspace*{0.1em}}}

% Common symbols for sets
\newcommand{\AAA}{\mathcal{A}}
\newcommand{\BBB}{\mathcal{B}}
\newcommand{\EEE}{\mathcal{E}}
\newcommand{\FFF}{\mathcal{F}}
\newcommand{\CCC}{\mathcal{C}}
\newcommand{\III}{\mathcal{I}}
\newcommand{\JJJ}{\mathcal{J}}
\newcommand{\RRR}{\mathcal{R}}
\newcommand{\LLL}{\mathcal{L}}
\newcommand{\KKK}{\mathcal{K}}
\newcommand{\PPP}{\mathcal{P}}
\newcommand{\VVV}{\mathcal{V}}
\newcommand{\SSS}{\mathcal{S}}
\newcommand{\XXX}{\mathcal{X}}
\newcommand{\ZZZ}{\mathcal{Z}}

\newcommand{\ADV}  {\mathsf{Adv}}
\newcommand{\Adv}  {\mathsf{Adv}^{\mathsf{DDH}}_{\group}}
\newcommand{\DL}  {\mathsf{Dlog}}


\newcommand{\Fvote}{\mathcal{F}_{\textsc{Vote}}}
\newcommand{\Fledger}{\mathcal{G}^{Q}_{\textsc{Ledger}}}
\newcommand{\Pivote}{\Pi_{\textsc{Vote}}}




\newcommand{\corrupt}  {\mathsf{cor}}

\newcommand{\readL}  {\textsc{Read}}
\newcommand{\readLR}  {\textsc{ReadReturn}}
\newcommand{\tamper}  {\textsc{Tamper}}
\newcommand{\tick}  {\textsc{tick}}
\newcommand{\post}  {\textsc{Post}}
\newcommand{\postN}  {\textsc{PostNotify}}

\newcommand{\honest}  {\mathsf{honest}}
\newcommand{\data}  {\mathsf{data}}
\newcommand{\buffer}  {\mathsf{buffer}}
\newcommand{\online}  {\mathsf{online}}
\newcommand{\Tally}  {\mathsf{TallyAlg}}
\newcommand{\leak}  {\textsc{Leak}}
\newcommand{\reveal}  {\textsc{Reveal}}
\newcommand{\init}  {\textsc{Init}}
\newcommand{\initN}  {\textsc{InitNotify}}
\newcommand{\vote}  {\textsc{Vote}}
\newcommand{\castN}  {\textsc{CastNotify}}
\newcommand{\delegate}  {\textsc{Delegate}}
\newcommand{\tally}  {\textsc{Tally}}
\newcommand{\tallyN}  {\textsc{TallyNotify}}
\newcommand{\readT}  {\textsc{ReadTally}}
\newcommand{\readTR}  {\textsc{ReadTallyReturn}}

\newcommand{\delegateN}  {\textsc{DelegateNotify}}
\newcommand{\voteN}  {\textsc{VoteNotify}}

\newcommand{\VC}{\mathsf{C}}
\newcommand{\EXP}{\mathsf{E}}
\newcommand{\Voter}{\mathsf{V}}

\newcommand{\PoK}{\mathsf{PoK}}

\newcommand{\opt}{\mathsf{opt}}

\newcommand{\Enc}{\mathsf{Enc}}
\newcommand{\Dec}{\mathsf{Dec}}
\newcommand{\SEnc}{\mathsf{SEnc}}
\newcommand{\SDec}{\mathsf{SDec}}
\newcommand{\HEnc}{\mathsf{HEnc}}
\newcommand{\HDec}{\mathsf{HDec}}
\newcommand{\group}{\mathbb{G}}
\newcommand{\pk}{\mathsf{pk}}
\newcommand{\sk}{\mathsf{sk}}

\newcommand{\ck}{\mathsf{ck}}
\newcommand{\symk}{\mathsf{k}}

\newcommand{\Gen}{\mathsf{Gen}_{\mathsf{gp}}}
\newcommand{\Genpkc}{\mathsf{Gen}}

\newcommand{\nizk}{\mathsf{NIZK}}

\newcommand{\Gencom}{\mathsf{KeyGen}^{\mathsf{C}}}
\newcommand{\Open}{\mathsf{Open}}
\newcommand{\Verify}{\mathsf{Verify}}

\newcommand{\DLEQProve}{\mathsf{DLEQProve}}
\newcommand{\DLEQVerify}{\mathsf{DLEQVerify}}
% \newcommand{\nizk}{\mathsf{NIZK}}
% \newcommand{\validate}{\mathsf{Validate}}
% \newcommand{\isvalidstate}{\mathsf{isvalidstate}}
% \newcommand{\valid}{\mathsf{valid}}
% \newcommand{\invalid}{\mathsf{invalid}}



\newenvironment{boxfig}[2]{%
	\begin{figure}[hbt!]
		\newcommand{\FigCaption}{#1}
		\newcommand{\FigLabel}{#2}
		%     \vspace{-0.25cm}
		\begin{center}
			\begin{small}
				\begin{tabular}{@{}|@{~~}l@{~~}|@{}}
					\hline
					\rule[-1.5ex]{0pt}{1ex}\begin{minipage}[b]{.95\linewidth}
						\vspace{1ex}
						\smallskip
					}{%
				\end{minipage}\\
				\hline
			\end{tabular}
		\end{small}
		%       \vspace{-0.25cm}
		\caption{\FigCaption}
		%       \figlab{\FigLabel}
	\end{center}
	\vspace{-0.5cm}
\end{figure}
}

\newcommand{\mybox}[5]{
    \begin{figure*}[hbt!]
        \centering
    \begin{tikzpicture}
        \node[anchor=text,text width=\columnwidth-1.2cm, draw, rounded corners, line width=1pt, fill=#3, inner sep=5mm] (big) {\\#4};
        \node[draw, rounded corners, line width=.5pt, fill=#2, anchor=west, xshift=5mm] (small) at (big.north west) {#1};
    \end{tikzpicture}
    \caption{#5}
    \end{figure*}
}

\newcommand{\hash}{\mathsf{hash}}
\newcommand{\valid}{\mathsf{valid}}
\renewcommand{\vec} [1] {\bold{#1}}

\newcommand{\myhalfbox}[5]{
    \begin{figure}[htpb]
        \centering
    \begin{tikzpicture}
        \node[anchor=text,text width=\columnwidth-1.2cm, draw, rounded corners, line width=1pt, fill=#3, inner sep=5mm] (big) {\\#4};
        \node[draw, rounded corners, line width=.5pt, fill=#2, anchor=west, xshift=5mm] (small) at (big.north west) {#1};
    \end{tikzpicture}
    \caption{#5}
    \end{figure}
}

\makeatletter
\setlength{\@fptop}{0pt}
\makeatother