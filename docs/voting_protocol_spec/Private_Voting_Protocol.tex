\section{Privacy-preserving Voting Protocol}

Even though the voting protocol described in the previous sections preserves privacy of a voter's choice, it does not preserve the privacy of a voting power (i.e., amount of stake) possesed by a voter.

In \cite{ZBO20} there was presented a non-trivial extension to the original protocol from \cite{ZOB18} that allows keep an amount of stake of voters (and respectively their voting power) in private. In this section we provide details of this extension.

In general, it involves modification of a voter ballot and some small modifications to the tally procedure, specifically how the voter ballots are summed up. Other parts of the protocol are not affected.

\subsection{Ballots casting}

Recall from the original protocol that a voter/expert choice is represented as a unit vector $e^{(l)}_i \in \{0,1\}^{l}$, where its $i$-th, $i \in [1,..,l]$, coordinate is $1$ and the rest coordinates are $0$.

The expert's choice  is represented by one of the unit vectors $(e^{(3)}_1, e^{(3)}_2, e^{(3)}_3)$, where $e^{(3)}_1$ stands for `Abstain', $e^{(3)}_2$ stands for `Yes', and $e^{(3)}_3$ stands for `No'.

The voter's choice is represented by the concatenation of two  unit vectors $( e^{(m)}_i , e^{(3)}_j )$, where $e^{(m)}_i$, $i\in[0,m]$ stands for the delegation choice ($m$ is the number of experts) and $e^{(3)}_i$, $i\in[0,2]$ stands for the voting choice.

Before publishing voter's/expert's choices on the blockchain they are encrypted. Let us denote a coordinate-wise encryption of $e^{(\ell)}_i$ as $\Enc_{\pk}(e^{(\ell)}_i )$, i.e. 
\[\Enc_{\pk}(e^{(\ell)}_i ) = \Enc_{\pk}(e^{(\ell)}_{i,1}), \ldots, \Enc_{\pk}(e^{(1)}_{i,\ell}),\]
where $e^{(\ell)}_i = (e^{(\ell)}_{i,1},\ldots, e^{(\ell)}_{i,\ell})$ and $\pk$ is a shared election public key generated during the DKG stage. Let us denote an encrypted unit vector as $u^{(l)}_i$, so that $u^{(l)}_{i}=\Enc_{\pk}(e^{(\ell)}_{i})$. In the original protocol, $u^{(l)}_i$, together with a proof of its correctness, constitutes a voter/expert ballot. Moreover, it is supposed that the amount of stake of a voter is publickly known.

In the extended protocol, a voter ballot\footnote{Note that the extension is applied only to voters ballots, experts ballots are unaffected} additionally includes a vector $v^{(l)}_i=(v^{(l)}_{i,1},...,v^{(l)}_{i,l})$, where each element is calculated in the following way:
\[v^{(l)}_{i,j}=\hat{\alpha}^{e_j}\cdot\Enc_{\pk}(0), \]
where $\hat{\alpha}=\Enc_{\pk}(\alpha)$ and $\alpha$ is an amount of stake of a voter (his voting power). Note that the vector $v^{(l)}_i$ is basically an encryption of a vector $\alpha \cdot e^{(l)}_i$ (an initial unit vector where each element is multiplied by $\alpha$).

Moreover, a voter adds an additional zero-knowledge proof of correct relation between vectors $u^{(l)}_i$, $v^{(l)}_i$ and $\hat\alpha$ (see \ref{fig:MultRelationNIZK}). It is supposed that $\hat\alpha$ is publicly available and it is externally validated that it encrypts  a correct amount of stake. In case of a blockchain system, validation of $\hat\alpha$ might require another additional proof that will ensure correctness of $\alpha$. This proof is specific to the platform where the protocol is used and does not considered in the current document.

The full protocol for ballot casting is depicted in Figure~\ref{fig:private_vote}.\footnote{See implementation here:\\ \href{https://github.com/input-output-hk/treasury-crypto/blob/master/src/main/scala/io/iohk/protocol/voting/Voter.scala}{https://github.com/input-output-hk/treasury-crypto/../protocol/voting/Voter.scala}}

\begin{boxfig}{\label{fig:private_vote}Ballots casting}{}
\footnotesize
	\textbf{Preparation phase:}
	\begin{itemize}
	    \item Retrieve shared election public key $\pk$ generated with $\Pi_{\textsc{DKG}}$ as desceibed in Figure~\ref{fig:DKG}.
	\end{itemize}	
	
	\textbf{Ballots casting phase:}
	\begin{itemize}
		\item Upon issuing a voting ballot, an expert $\EXP_i$ does the following:
		\begin{itemize}
			\item For each submitted proposal $p_k \in \mathcal{P}$: 
			\begin{itemize}
			    \item create a unit vector $e^{(3)}_{k,\ell}$ according to his choice (e.g, $e^{(3)}_{k,1}$ for "Abstain", $e^{(3)}_{k,2}$ for "Yes", and $e^{(3)}_{k,3}$ for "No");
			    \item pick randomness $r_{k,1},r_{k,2},r_{k,3}\leftarrow \ZZ_p$ and compute $c_{k,t}\leftarrow \Enc_{\pk}(e^{(3)}_{k,t}; r_{k,t})$, $t\in[3]$;
			    \item produce a unit vector proof $\pi_k$ showing that $\{c_{k,t}\}_{t\in[3]}$ encrypts a unit vector\footnote{The proof was described in section [\ref{sec:SHVZK}]}.
			\end{itemize}
			  
			\item Send a transaction with $(Ballot, (\EXP_i, \{\{c_{k,t}\}_{t\in[3]}, \pi_k\}_{k \in |\mathcal{P}|}))$ to the blockchain.
		\end{itemize}	
		
		\item Upon issuing a voting ballot, a voter $\Voter_i$ does the following:
		\begin{itemize}
			\item For each submitted proposal $p_k \in \mathcal{P}$: 
			\begin{itemize}
			    \item create a unit vector $e^{(m+3)}_{k,\ell}$ so that:
			    \begin{itemize}
			        \item if $\Voter_i$ wants to delegate, then $e^{(m+3)}_{k,\ell}:= (e^{(m)}_{k,i}, e^{(3)}_0)$, where $i \in [1,m]$ is an index of an expert;
			        \item otherwise, if $\Voter_i$ wants to vote directly, then $e^{(m+3)}_{k,\ell}:= (e^{(m)}_0, e^{(3)}_{k,i})$, where $i \in \{1,2,3\}$ depends on the choice (Abstain, Yes, or No correspondingly);
			    \end{itemize}

			    \item pick randomness $r_{k,1},\ldots, r_{k,m+3}\leftarrow \ZZ_q$ and compute $u_{k,t}\leftarrow \Enc_{\pk}(e_t; r_{k,t})$, $t\in[m+3]$, where $e_t$ is a $t$-th bit of the vector $e^{(m+3)}_{k,l}$;
			    \item produce a proof $\pi_k^u$ showing that $\{u_{k,t}\}_{t\in[m+3]}$ encrypts a unit vector (see [\ref{fig:unit_zk_prover}]);
			    \item pick randomness $r_{k,\alpha}\leftarrow \ZZ_q$ and compute $\hat\alpha=\Enc_{\pk}(\alpha,r_{k,\alpha})$, where $\alpha$ is an amount of stake of a voter;
			    \item pick randomness $z_{k,1},\ldots, z_{k,m+3}\leftarrow \ZZ_q$ and compute $v_{k,t}\leftarrow \hat\alpha^{e_{t}} \cdot \Enc_{\pk}(0;z_{k,t})$, $t\in[m+3]$, where $e_{t}$ is a $t$-th bit of the vector $e^{(m+3)}_{k,l}$;
			    \item produce a proof $\pi_k^v$ showing that $\{v_{k,t}\}_{t\in[m+3]}$ is in correct relation to $\{u_{k,t}\}_{t\in[m+3]}$ and $\hat\alpha$ (see [\ref{fig:MultRelationNIZK}]);
			\end{itemize}
			  
			\item Send a transaction with $(Ballot, (\Voter_i, \hat\alpha,\{\{u_{k,t},v_{k,t}\}_{t\in[m+3]}, \pi_k^u, \pi_k^v\}_{k \in |\mathcal{P}|}))$ to the blockchain.
		\end{itemize}	
	\end{itemize}
\end{boxfig}

\mybox{Vector multiplicative relation ZK argument}{white!40}{white!10}{
\textbf{Statement:} $\pk$, $C$, $\{A_i:=  \Enc_{\pk}(e_{i};r_i)\}_{i=0}^{n-1}$ and $\{V_i := C^{e_i} \cdot \Enc_{\pk}(0;t_i)\}_{i=0}^{n-1}$\\
\textbf{Witness:} $\{e_i,r_i,t_i\}_{i=0}^{n-1}$
\\~\\
\textbf{Prover:}
\begin{itemize}
	\item Pick random $x,y,z\leftarrow \ZZ_p$;
	\item Compute $X:=\Enc_{\pk}(x,y)$ and $Z:=C^x\cdot \Enc_{\pk}(0; z)$;
	\item Compute a challenge $\rho=hash(\pk\ |\ C\ |\ X\ |\ Z)$;
	\item Compute: 
	\begin{itemize}
		\item $x':= x+ \sum_{i=0}^{n-1} e_i \cdot \rho ^ {i+1}$;
		\item $y':= y+ \sum_{i=0}^{n-1} r_i \cdot \rho ^ {i+1}$;
		\item $z':= z+ \sum_{i=0}^{n-1} t_i \cdot \rho ^ {i+1}$;
	\end{itemize}
	\item Return proof $\pi:=(X, Y, x',y',z')$.
\end{itemize}

\textbf{Verifier:}
\begin{itemize}
	\item Compute a challenge $\rho=hash(\pk\ |\ C\ |\ X\ |\ Z)$;
	\item Return $\mathsf{valid}$ if and only if the following is true:
	\begin{itemize}
	\item $Z \cdot \prod_{i=0}^{n-1}V_i^{\rho^{i+1}} = C^{x'} \cdot \Enc_{\pk}(0; z')$
	\item $X\cdot \prod_{i=0}^{n-1} A_i^{\rho^{i+1}} = \Enc_{\pk}(x',y')$
	\end{itemize}
\end{itemize}	
}{\label{fig:vec_zk} Vector multiplicative relation ZK argument}